\section{Approach}
\subsection{Version 1}
\begin{itemize}
    \item[] Implementation 
        \begin{itemize}
            \item[] \textbf{Ist Phase} Preprocessing \\ Creating signature using linked list and cross comparison over set of sequence of system calls.
            \item[] \textbf{IInd Phase} Processing \\ Detecting malware by comparisons of set of signature and input sequence of system calls.
        \end{itemize}
    \item[] {Advantage}
        \begin{itemize}
            \item[*] Signature was always linear
            \item[*] Signature required less space
            \item[*] Easier computation and used link list
        \end{itemize}
    \item[] {Disadvantage}
        \begin{itemize}     
            \item[*] Signature followed the smallest input
            \item[*] Signature converged became inadequate as it was too small.
            \item[*] Did not consider the system calls that are rejected while cross comparing if it didn't exist in both.
            \item[*] Huge chances of False positive hence to modify the preliminary checks.
        \end{itemize}
    \item[] {NOTE:}
        \begin{itemize}                                        
            \item[] Signature needs to include branches not done during learning phase.Signature needs to incorporate a feedback system so as to develop effective system. Information loss over space complexity will lead to a poor system. All information should be captured during the learning phase. System calls like mutex, poll and futex repeated many times hence were not considered during signature creation.
        \end{itemize}
\end{itemize}
\subsection{Version 2}
\begin{itemize}
    \item[] {New signature design:}
        \begin{itemize}
            \item[*] A backbone sequence is created by finding the common sequence of system calls in the input dataset.
            \item[*] Rest of the elements are added to the signature in the forms of pairs as per their occurrences in the input sequence after processing.
            \item[*] Unique list of system calls to identify those system calls that are never used.
        \end{itemize}
\end{itemize}
