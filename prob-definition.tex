\section{Problem Definition}

\textsf{To design and implement an Intrusion Detection System that tracks the system calls of an application so as to detect a malware.} \\

\section{Introducion}
An Intrusion Detection System or IDS in short are applications that identifies attacks on a system or network and in turn alerts the administrator. There are many types of classifications of IDS. On the basis of techniques used by the IDS it is classified as anomaly based detection and signature based detection. The anomaly based detection system uses normal working of system as stable state and any abnormalities to this state is detected as an attack whereas signature based IDS uses signature of attacks to find whether a particular attack has been devised on the system. Anomaly based systems can detect new attack but at the same time have huge probabilities of false positives. A signature based system solves this but fail to detect new attacks. IDS are also classified as host based IDS which monitors a single system and network based IDS which monitors a whole network. \\ \\
All applications when required communicate with the kernel for all its services for both hardware and software. Whenever an application requires the intervention of the kernel it requests a software interrupt to get the attention of the kernel. These software interrupts are called system calls and are predefined in an Operating System. The fundamental idea of the project is to use these systems calls to detect attacks as all the activities require system calls. In an application the system call sequence can be depicted using a direct acyclic graph. A graph with directed edges and with no cycle in it is called a Directed Acyclic Graph or DAG. They could be used to show the dependency of a node on another node. Here in this context, we say that when a application is running, one set of system calls is dependent on another set of system calls, and this is used as a signature for that process. \\ \\ 
We attempt to dynamically find the malware or vulnerability of a program by analyzing the deviations, if any, from the combined signature generated by considering all the processes of an application.



%% \begin{quotation}
%%     \hspace{100pt} \textbf{Abstract} \\
%% \textsf{In this project we develop a method for detecting malware using the sequence of system calls invoked by a process. This works as a host based Intrusion Detection System by comparing the process with a signature database. Each signature has a primary backbone that represents the workflow of an application and a secondary list of possible branches from the main thread. In the preprocessing stage, the signature backbone is created using the largest common subsequence derived from the input dataset. The differences from the backbone are saved as branches in the secondary list. It uses an idea similar to that of poset to represent the branches for the signature. The attack is identified by measuring the deviation of system calls of a suspected application from the signature either in a dynamic or static environment.}
%% \end{quotation}
